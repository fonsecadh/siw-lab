\documentclass[11pt]{article}
\usepackage[T1]{fontenc}
\usepackage{lmodern}
\usepackage{parskip}
\usepackage[colorlinks=true,urlcolor=Blue,linkcolor=black,citecolor=black]{hyperref}
\usepackage{graphicx}
\usepackage{amsmath}
\usepackage[utf8]{inputenc}
\usepackage[spanish]{babel}
\usepackage{fancyhdr}
\usepackage{csquotes}
\usepackage{lastpage}
\usepackage{array}
\usepackage{listings}
\usepackage{color}
\definecolor{dkgreen}{rgb}{0,0.6,0}
\definecolor{gray}{rgb}{0.5,0.5,0.5}
\definecolor{mauve}{rgb}{0.58,0,0.82}
\usepackage[affil-it]{authblk}
\usepackage[activate={true,nocompatibility},final,tracking=true,kerning=true,spacing=true,factor=1100,stretch=10,shrink=10]{microtype}
\usepackage[hmargin=2cm,top=4cm,headheight=65pt,footskip=65pt]{geometry}

% Para el código de Python
\lstset{frame=tb,
  language=Python,
  aboveskip=3mm,
  belowskip=3mm,
  showstringspaces=false,
  columns=flexible,
  basicstyle={\small\ttfamily},
  numbers=none,
  numberstyle=\tiny\color{gray},
  keywordstyle=\color{blue},
  commentstyle=\color{dkgreen},
  stringstyle=\color{mauve},
  breaklines=true,
  breakatwhitespace=true,
  tabsize=3
}

% Documento
\begin{document}
% Título
\title{SIW PL7. PageRank.}
\author{Hugo Fonseca Díaz\\ \email{uo258318@uniovi.es}}
\affil{Escuela de Ingeniería Informática. Universidad de Oviedo.}
\maketitle
% Requerimientos
\section{Requerimientos}
Los requerimientos para el entorno de ejecución de esta práctica son:
\begin{itemize}
    \item Python 3.8
    \item Biblioteca \textit{numpy}
\end{itemize}
% Detalles de la implementación
\section{Detalles de la documentación}
Para esta práctica se nos pedía realizar una implementación del algoritmo PageRank ideado por Larry Page y Sergey Brin. Dicho algoritmo está implementado en el script de python \textit{graph.py}. En él se encuentra la clase \textit{Graph}, que modela un grafo. Dicha clase es la encargada de convertir los bordes pasados como argumento en su constructor a una matriz de adyacencia. Dicha matriz sera normalizada y convertida en una matriz estocástica que será la matriz con la que trabajará el algoritmo. Además, en el constructor se definirá un diccionario \textit{PR} que guardará el nombre de los nodos junto a su valor de PageRank. A continuación hablaremos sobre el método \textit{page rank} más en detalle.

El método \textit{page rank} es el siguiente:

\begin{lstlisting}
def page_rank(self, damping = 0.85, limit = 1.0e-8):
    M = self.__stochastic__(self.M) # Stochastic matrix
    N = M.shape[1] # Number of nodes
    R = np.full((1, N), (1 / N)) # Solution eigenvector
    R = R / np.linalg.norm(R, 1)
    M_hat = (1 - damping) / N + damping * M # Matrix for calculating PR
    while True: # Do while the result is over the limit
        prev_R = R
        R = np.dot(R, M_hat) # Recalculate solution eigenvector
        if np.linalg.norm(R - prev_R, 2) <= limit:
            break
    # Dictionary with the PR values
    # key = node, value = PR(node)
    pos = 0
    for n in self.PR.keys():
        self.PR[n] = R[0, pos] # Insert eigenvectors into PR dictionary
        pos += 1
    return self.PR
\end{lstlisting}

Para entenderlo mejor vamos a hablar sobre las variables utilizadas:
\begin{itemize}
    \item \textbf{$\boldsymbol{M}$:} matriz de adyacencia. Normalizada y estocástica.
    \item \textbf{$\boldsymbol{N}$:} número de nodos en el grafo.
    \item \textbf{$\boldsymbol{R}$:} \textit{eigenvector} solución con los valores del PageRank.
    \item \textbf{$\boldsymbol{M_{hat}}$:} Matriz con la que calculamos los valores de $R$ en cada iteración.
    \item \textbf{$\boldsymbol{PR}$:} diccionario cuyas claves son los nodos del grafo y cuyos valores son el PageRank asociado a cada uno de esos nodos.
\end{itemize}

El grueso del algoritmo está inspirado en el código de Python ofrecido por \textit{Wikipedia} en \cite{codewikipedia}.
% Bibliografía
\begin{thebibliography}{8}
\bibitem{codewikipedia}
Entrada de la Wikipedia (Inglés) para el algoritmo \textit{PageRank}, \url{https://en.wikipedia.org/wiki/PageRank#Python}. Última vez accedido 5 Noviembre 2020.
\end{thebibliography}
\end{document}

